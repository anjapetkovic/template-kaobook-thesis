%% Input and output encoding:
\usepackage[T1]{fontenc}
\usepackage[utf8]{inputenc}  % to allow unicode in source

%% Language
\usepackage[english]{babel}

\usepackage{fontspec}

% Load the bibliography package
\usepackage[maxbibnames=99]{styles/kaobiblio}
\addbibresource{thesis.bib} % Bibliography file

% Load mathematical packages for theorems and related environments. NOTE: choose only one between 'mdftheorems' and 'plaintheorems'.
\usepackage{styles/mdftheorems}
%\usepackage{styles/plaintheorems}

% \graphicspath{{img/}} % Paths in which to look for images

\makeindex[columns=3, title=Alphabetical Index, intoc] % Make LaTeX produce the files required to compile the index

\makeglossaries % Make LaTeX produce the files required to compile the glossary

\makenomenclature % Make LaTeX produce the files required to compile the nomenclature


% Customising margin citations
\renewcommand{\formatmargincitation}[1]{%
	\color{Gray!80} \parencite{#1}: \citeauthor*{#1}
	\sbox0{\citeyear{#1}}%
    \ifdim\wd0=0pt
      {}% if is empty
    \else
      {(\citeyear{#1}), }% if is not empty
    \fi
	\citetitle{#1}\\%
}

%% AMS and other general math packages:
\usepackage{amsfonts}
\usepackage{amsmath}           % basic ams math environments and symbols
\usepackage{amssymb,amsthm}    % ams symbols and theorems

\usepackage{mathtools} % for \mathrlap
\usepackage{upgreek} % for upright greek letters
% \usepackage{newtxtext,newtxmath}

%% For slightly smaller tokens for syntax classes
\usepackage{scalerel}

%% For nicely framed figure
\usepackage{mdframed}

\usepackage{xcolor}

%% For syntax of type theory:
\usepackage{mathpartir} % for inference rules


%% For code coloring
\usepackage{listings} % package for code
\usepackage{lstcoq} % Coq code coloring

