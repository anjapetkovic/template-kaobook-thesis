\chapter{Abstract}

In this thesis we present a meta-analysis of a wide class of general type theories, focusing on three aspects: transformations of type theories, elaboration of type theories, and a general equality checking algorithm.

Type theories provide the mathematical foundations of many proof assistants. We build towards understanding of how they interact by studying their meta-theoretic properties and checking them against an implementation of the flexible proof assistant Andromeda~2, which supports user-specified type theories.

Our meta-analysis is built on the definition of finitary type theories. We define \emph{syntactic transformations} of type theories and prove they form a relative monad for the syntax. To account for the derivability structure, we upgrade the definition to \emph{type-theoretic transformations} that cover some familiar examples, like propositions as types translation and the definitional extension. Once these definitions are accomplished we prove some meta-theorems.
%
The usefulness of type-theoretic transformations is unveiled in the definition of an elaboration and we prove an elaboration theorem, saying that every finitary type theory has an elaboration.

To tackle the implementational side, we design a general and user-extensible equality checking algorithm, applicable to a finitary type theories. The algorithm is composed of a type-directed phase for applying extensionality rules and a normalization phase based on computation rules. Both kinds of rules are defined using the type-theoretic concept of object-invertible rules. We specify sufficient syntactic criteria for recognizing such rules and a simple pattern-matching algorithm for applying them.
A third component of the algorithm is a suitable notion of principal arguments, which determines a notion of normal form. By varying these, we obtain known notions, such as weak head-normal and strong normal forms.
%
We prove that our algorithm is sound.
%
We implemented it in the Andromeda~2 proof assistant.

\vspace{\baselineskip}

%\textbf{Math. Subj. Class.}
\noindent\textbf{\textup{2020} Mathematics Subject Classification:}
%% Type theory:
  03B38%
%% Mechanization of proofs and logical operations
% , 03B35%
%% Logic in computer science:
, 03B70%
%% Categorical logic, topoi [See also 18B25, 18C05, 18C10]
% , 03G30%
%% Categories of fibrations, relations to K-theory, relations to type theory:
% , 18N45%
%% Theories (e.g., algebraic theories), structure, and semantics
, 18C10%
%% Categorical semantics of formal languages
% , 18C50%
%% Software, source code, etc. for problems pertaining to computer science
%, 68-04%
%% Theory of programming languages
% , 68N15%
%% Functional programming and lambda calculus
% , 68N18%
%% Modes of computation (nondeterministic, parallel, interactive, probabilistic, etc.)
% , 68Q10%
%% Semantics in the theory of computing [See also 03B70, 06B35, 18C50]
% , 68Q55%
%% Theorem proving (automated and interactive theorem provers, deduction, resolution, etc.):
, 68V15%
%% Formalization of mathematics in connection with theorem provers
, 68V20%
%% Symbolic computation and algebraic computation
% , 68W30%
%% Metamathematics of constructive systems:
, 03F50%
%% Proof theory, general (including proof-theoretic semantics)
, 03F03%
%% Cut-elimination and normal-form theorems
%, 03F05%
%% Structure of proofs
, 03F07


\noindent\textbf{Keywords:}
Dependent type theory,
algebraic theory,
%categorical logic,
proof assistant,
type-theoretic elaboration
equality checking
