\chapter{Introduction}

\begin{definition}
    \labdef{TT-transformation}
    A \defemph{type-theoretic transformation} $f \colon \theory{T} \to \theory{U}$ is a syntactic transformation $f \colon \Sigma_\theory{T} \to \Sigma_\theory{U}$ such that for each rule $\rawRule{\Theta}{\J}$ in $\theory{T}$ we have a derivation $\derivation$ of $\rawRule{\act{f} \Theta}{\act{f}\J}$ in $\theory{U}$.
\end{definition}

\begin{theorem}[The Elaboration Theorem]
    \labthm{elaboration}
    Every finitary type theory has an elaboration.
\end{theorem}

\begin{lemma}
    \lablemma{action-inter-substitution}
    Action interacts with substitution:
    Let $f \colon \Sigma_1 \to \Sigma_2$ be a syntactic transformation and $e$ and $t$ well-formed expressions in the syntax over $\Sigma_1$. The following equation holds
    %
    \begin{equation*}
        \act{f}(e[t/x]) = (\act{f}e)[\act{f}t/x].
    \end{equation*}
    %
\end{lemma}

  \begin{proposition}[Weakening]
    \labprop{tt-weakening}
    For a raw type theory:
    %
    \begin{enumerate}
    \item If $\Theta; \Gamma_1, \Gamma_2 \types \JJ$ then $\Theta; \Gamma_1, \var{a} \of A, \Gamma_2 \types \JJ$.
    \item If $\Theta_1, \Theta_2; \Gamma \types \JJ$ then $\Theta_1, \symM \of \BB, \Theta_2; \Gamma \types \JJ$.
    \end{enumerate}
    %
    An analogous statement holds for boundaries.
  \end{proposition}

  \marginnote[-1cm]{Note that since syntactic transformations preserve classes, propositions have to be mapped to terms and not syntactic types.}
\begin{example}
    \labexample{propositions-as-types}
    One of the most well-known transformations is the \defemph{propositions as types}, also known as the Curry-Howard correspondence~\sidecite[]{Wadler15:prop-as-types, Curry34:prop-as-types, Curry-Freys58:combinatoryLogic, Howard80:Formulae-as-types} that translates mathematical proofs of first-order logic to terms of type theory and thus gives them computational content.
    We exhibit the transformation from first-order logic (FOL) to the relevant fragment of the Martin-Löf type theory (MLTT) \sidecite[+2cm]{Martin-Lof-1972,Martin-Lof-1973,Martin-Lof-1979,martin-lof:bibliopolis} with one universe of (small) types.

    The full formulation of FOL and MLTT, as well as the type-theoretic transformation is in the Appendix~\refch{propositions-as-types}.
    The idea is to map the type $\sym{o}$ of propositions in FOL to the universe $\sym{U}$ of MLTT and the terms that represent propositions $\sym{p} : \sym{o}$ to the codes
    for types $\sym{a} : \sym{U}$. We also map the type $\sym{i}$ of individuals in FOL to the $\sym{base}$ type in MLTT.

    The rest of the connectives are mapped as usual, for example the conjunctions $\sym{conj}$ are mapped to the simple products $\upsigma(\sym{p}, \abstr{x} \sym{q})$ and universal quantifiers to the dependent products.

    In~\refch{propositions-as-types} we also write justification that this is indeed a type-theoretic transformation,~i.e. that derivability is preserved.
\end{example}


\begin{figure}[pht]
    \centering
    \small
    \begin{ruleframe}
    \begin{mathpar}
    \inferenceRule{TT-Ty-Refl}
    { \Theta; \Gamma \types \isType { A } }
    { \Theta; \Gamma \types A \equiv A }

    \inferenceRule{TT-Ty-Sym}
    { \Theta; \Gamma \types A \equiv B }
    { \Theta; \Gamma \types B \equiv A }

    \inferenceRule{TT-Ty-Tran}
    { \Theta; \Gamma \types A \equiv B \\
      \Theta; \Gamma \types B \equiv C }
    { \Theta; \Gamma \types A \equiv C }

    \inferenceRule{TT-Tm-Refl}
    { \Theta; \Gamma \types t : A }
    { \Theta; \Gamma \types t \equiv t : A }

    \inferenceRule{TT-Tm-Sym}
    { \Theta; \Gamma \types s \equiv t : A }
    { \Theta; \Gamma \types t \equiv s : A}

    \inferenceRule{TT-Tm-Tran}
    { \Theta; \Gamma \types s \equiv t : A \\
      \Theta; \Gamma \types t \equiv u : A }
    { \Theta; \Gamma \types s \equiv u : A }

    \inferenceRule{TT-Conv-Tm}
    { \Theta; \Gamma \types t : A \\
      \Theta; \Gamma \types A \equiv B }
    { \Theta; \Gamma \types t : B }

    \inferenceRule{TT-Conv-Eq}
    {
      \Theta; \Gamma \types s \equiv t : A \\
      \Theta; \Gamma \types A \equiv B }
    { \Theta; \Gamma \types s \equiv t : B }
    \end{mathpar}
    \end{ruleframe}
    \caption{Equality closure rules.}
    \labfig{equality-rules}
  \end{figure}