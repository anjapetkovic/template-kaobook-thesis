% -------------------------------------------------------------
% Kaobook side notes
% -------------------------------------------------------------

% Reference to bibliography
\newcommand{\refbib}{\hyperref[bib]{Bibliography}}

% Reminder box
\newcommand{\reminder}[3][0pt]{%
  \marginnote[#1]{
    \begin{kaobox}[
      frametitle=Reminder: #2,
      backgroundcolor=YellowGreen!25!White,
      frametitlebackgroundcolor=YellowGreen!25!White
    ]
      #3
    \end{kaobox}
  }
}

% Definition box
\newcommand{\sidedef}[3][0pt]{%
  \marginnote[#1]{
    \begin{kaobox}[
      % frametitle=Definition: #2,
      backgroundcolor=GreenYellow!25!White,
      % frametitlebackgroundcolor=GreenYellow!25!White
    ]
      #3
    \end{kaobox}
  }
}

\newcommand{\sidedeffr}[3][0pt]{%
  \marginnote[#1]{
    \begin{kaobox}[
      frametitle=Définition: #2,
      backgroundcolor=GreenYellow!25!White,
      frametitlebackgroundcolor=GreenYellow!25!White
    ]
      #3
    \end{kaobox}
  }
}

% Warning box
\newcommand{\warn}[1]{%
  \begin{kaobox}[
    % frametitle=,
    backgroundcolor=OrangeRed!25!White,
    % frametitlebackgroundcolor=GreenYellow!25!White
  ]
    #1
  \end{kaobox}
}


% -------------------------------------------------------------

\newcommand{\defemph}[1]{\textcolor{Maroon}{\textbf{\emph{#1}}}} % defined term

% -------------------------------------------------------------
% Type theoretic symbols
% -------------------------------------------------------------

\newcommand{\dummy}{{\star}} % dummy value
\newcommand{\mto}{{\mapsto}} % short \mapsto (can you find a shorter arrow?)
\newcommand{\finmap}[1]{\langle #1 \rangle} % finite map

\newcommand{\Inst}[4]{#1 \in \mathrm{Inst}(#2, #3, #4)} % \Inst{I}{\Xi}{\Theta}{\Gamma} is an instantiation of \Xi over \Theta and \Gamma
\newcommand{\upto}[2]{#1_{(#2)}} % restriction of an instantiation
\newcommand{\act}[1]{#1_*} % action of an instantiation
\newcommand{\upact}[2]{#1_{(#2)*}} % combination of restriction and action



\newcommand{\theory}[1]{\mathcal{#1}} % a type theory
\newcommand{\derivation}{\mathcal{D}}

% BNF grammars
\newcommand{\bnfis}{\mathrel{\;{:}{:}{=}\ }}
\newcommand{\bnfor}{\mathrel{\;\big|\ \ }}

% Frame around a set of rules, to be used in  figures
\newenvironment{ruleframe}{\begin{mdframed}[linecolor=gray]}{\end{mdframed}}

% Signatures

\newcommand{\sym}[1]{\mathsf{#1}} % a symbol
\newcommand{\symM}{\sym{M}} % generic meta-variable
\newcommand{\rawRule}[2]{#1 \Longrightarrow #2} % the raw rule

% The syntax of computations
\newcommand{\abstr}[1]{\{#1\}} % abstraction

\newcommand{\B}{\mathcal{b}} % generic boundary thesis
\newcommand{\BB}{\mathcal{B}} % abstracted boundary thesis

\newcommand{\J}{\mathcal{j}} % generic judgement thesis
\newcommand{\JJ}{\mathcal{J}} % abstracted judgement thesis

\newcommand{\var}[1]{\mathsf{#1}} % a font for free variables/atoms
\newcommand{\isType}[1]{#1\;\mathsf{type}}
\newcommand{\of}{{:}} % a : A with less spacing around the colon

\newcommand{\types}{\vdash} % another name for entailment
\newcommand{\ctx}{\;\mathsf{ctx}} % judgement that something is a valid context

\newcommand{\plug}[2]{#1{\setlength{\fboxrule}{0.5pt}\setlength{\fboxsep}{1pt}\fbox{\vphantom{$#1$}$#2$}}} % fill the head of a boundary


% rule name for use in text
\newcommand{\rulename}[1]{\textnormal{\textsc{#1}}}
\newcommand{\rref}[1]{\hyperref[#1]{\rulename{#1}}} % hyperlink a rule


% named inference rule
\definecolor{rulenameColor}{rgb}{0.5,0.5,0.5}
\newcommand{\inferenceRule}[3]{\inferrule*[lab={\label{#1}\rulename{\color{rulenameColor}#1}}]{#2}{#3}}
